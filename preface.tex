\chapter{序}

\begin{quotation}
满纸荒唐言,一把辛酸泪!都云作者痴,谁解其中味?\footnote{在人生迷茫时刻竟耗时耗力地整理笔记,若非痴傻定是疯癫。罢了,罢了!就按黄老师的模版慢慢写吧。}
\begin{flushright}
---~曹雪芹
\end{flushright}
\end{quotation}

我这个人做事情就是三天的热乎劲!真不知道这份笔记能完成到什么程度。不管结果如何,至少在此时此刻还是信心满满的。其实心中一直有个想法就是把经历的事情按照一个时间轴的形式记录下来---有时候看看自己以前写的日记觉得蛮有意思的---也许,多年后的某一天有空回头看看的话也不会觉得空虚!

开始我选择通过写~blog~的形式记录,而写的这些东西有不想被很多人看见于是将文章的属性改为“仅自己可见”,等到文章达到一定数量后才发现~blog~文章列表上仅全部都带一个锁行标志,仔细想想我已经违背了~blog~共享的初衷就放弃了。随后我选择邮箱中的私人记事本作为自己的私密领地,可渐渐的就不想登陆自己的邮箱了,而且记录的东西毫无排版美感之说。在这个几乎没有什么个人隐私而言的社会,自己记录点心事想法似乎是一种享受,是一种快乐!于是也不想再在选择记录方式上做太多的纠缠,就敲定用~\LaTeX~ 编写日记,用~git~进行管理\footnote{目前这两个工具用的都还不是很熟,就凑合着用吧!}。

目前估计不会对记录的内容做过多的分类,一些学习心得、笔记之类的东西可能会在一些不合适的章节中出现。不过我的长远目标是将生活学习分类,然后在各个子类中递归细分,使记录的东西有趣实用且易于查找。

本来是私人的东西,我想如果坚持写下去以后肯定会有人看到的,尤其是学习和技术部分,不过鄙人才疏学浅,恐有不足之处,还望各位高人斧正!

终了,就以一句警示结束这毫无文采的开头吧!
\begin{quotation}
“来时豪情万丈,走时空空行囊"\footnote{这句话是我高中二年级的语文老师赵勇说的,不知道会成为多少人的写照啊!}
\begin{flushright}
\vspace{40mm}
2013年5月16日\footnote{写到此时听见外头有布谷鸟的叫声,一看时间已是凌晨}
\end{flushright}
\end{quotation}

