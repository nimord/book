\documentclass[a4paper]{ctexart}

% 导言区,加载宏包和各项设置
\usepackage[hmargin=1.25in,vmargin=1in]{geometry}  %  word 版式
\usepackage[x11names]{xcolor} %must before tikz, x11names defines RoyalBlue3
\usepackage[bookmarksnumbered,unicode, pdfborder=1,breaklinks,colorlinks,linkcolor=RoyalBlue3,urlcolor=blue]{hyperref}
\usepackage{graphicx} % 图片
\usepackage{fancyhdr} % 页眉页脚
\usepackage[raggedright]{titlesec} % 标题
%\usepackage{enumitem}
\usepackage{listings}
\usepackage{tikz}

% 环境设置
\graphicspath{{figures/}} % 设置图片路径

% 页眉页脚格式
\pagestyle{fancy}
\fancyhf{}
\cfoot{\thepage}
\rhead{使用~Lorax~构建~DVD~安装小系统}
\lhead{\includegraphics[width=3.5cm]{logo/cs2c-logo}}
\renewcommand{\headrulewidth}{0.4pt}

% 行距
\linespread{1.6}

\lstset{
basicstyle=\ttfamily,
columns=flexible,
numbers=left,
numberstyle=\footnotesize,
keywordstyle=\color{blue!70},
commentstyle=\color{red!50!green!50!blue!50},
frame=single,backgroundcolor=\color{white}, 
framexleftmargin=2em,
breaklines=true,
breakautoindent=true,
breakindent=0em,
escapeinside=``,
xleftmargin=3.5em,
xrightmargin=1em, 
aboveskip=1em
}

\begin{document}
%%%% cover %%%%
\thispagestyle{empty}

\noindent\begin{minipage}{\textwidth}
	\begin{flushleft}
	\includegraphics{logo/cs2c-short-logo}
	\end{flushleft}
\end{minipage}

\vspace{\stretch{1}}
\noindent\begin{minipage}{\textwidth}
\centering
{\LARGE \bfseries 使用~Lorax~构建~DVD~安装小系统(NKWin)}
\noindent\rule[1.5ex]{\textwidth}{1pt}
\end{minipage}

\vspace{\stretch{2}}
\noindent\begin{center}
2016~年~07~月~15~日
\end{center}
\clearpage

% main 
\section{系统环境}
构建环境的主机系统尽量为中标麒麟操作系统或者~Fedora~系统,并且尽量使用~64~位版本\footnote{目前~64~位的基础系统可以构建~32~位及~64~位的安装小系统,但~32~位的基础系统仅能构建~32~位的安装小系统}。

\section{系统构建}
\subsection{安装~lorax~}
调整系统仓库源地址,安装~lorax rpm~包,一般情况下使用与~distribution~对应的版本。

lorax~安装完成后,可以查看~/usr/share/lorax~目录,在该目录下有相关的~tmpl~文件,\emph{通过调整这些文件可以控制安装小系统中集成的文件内容}。

\subsection{开始构建}
执行以下命令:
\begin{lstlisting}[language=bash]
lorax -p "NeoKylin Linux Desktop" -v "6.0" -r "NeoKylin Linux Desktop 6.0" --isfinal --volid="NeoKylin Linux Desktop 6.0" -s "REPO-URL" DEST-DIR
\end{lstlisting}

REPO-URL:构建安装小系统使用的仓库源地址(也可以使用本地建立的仓库源);

DEST-DIR:构建完成的安装系统文件存放位置。

以上命令顺利执行完成后,会在~DEST-DIR~目录下生成~DVD~版本所需要的安装小系统文件,然后加入~Packages~相关文件并生成~repodata~,最后压制成镜像文件即可。

在执行时可能遇到以下的问题:
\begin{lstlisting}
checking package signatures
running test transaction
running transaction
Traceback (most recent call last):
File "/usr/lib/python2.7/site-packages/yum/rpmtrans.py", line 458, in callback
File "/usr/lib/python2.7/site-packages/yum/rpmtrans.py", line 541, in _instProgress
File "/usr/lib/python2.7/site-packages/pylorax/yumhelper.py", line 102, in event
UnicodeEncodeError: 'ascii' codec can't encode characters in position 0-3: ordinal not in range(128)
FATAL ERROR: python callback <bound method RPMTransaction.callback of <yum.rpmtrans.RPMTransaction instance at 0x979e86c>> failed, aborting! 
\end{lstlisting}

解决方法:修正~/usr/lib/python2.7/site-packages/yum/rpmtrans.py~在~class RPMTransaction~中的~\verb|__init__|~中加入以下代码,修改默认的编码格式即可。

\begin{lstlisting}
reload(sys)
sys.setdefaultencoding("utf-8")
\end{lstlisting}


\end{document}