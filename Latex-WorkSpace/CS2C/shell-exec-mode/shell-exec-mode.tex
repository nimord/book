\documentclass[a4paper]{ctexart}

% 导言区,加载宏包和各项设置
\usepackage[hmargin=1.25in,vmargin=1in]{geometry}  %  word 版式
\usepackage[x11names]{xcolor} %must before tikz, x11names defines RoyalBlue3
\usepackage[bookmarksnumbered,unicode, pdfborder=1,breaklinks,colorlinks,linkcolor=RoyalBlue3,urlcolor=blue]{hyperref}
\usepackage{graphicx} % 图片
\usepackage{fancyhdr} % 页眉页脚
\usepackage[raggedright]{titlesec} % 标题
\usepackage{listings}

% 环境设置
\graphicspath{{figures/}} % 设置图片路径

% 页眉页脚格式
\pagestyle{fancy}
\fancyhf{}
\cfoot{\thepage}
\rhead{Shell~脚本执行的几种方式}
\lhead{\includegraphics[width=3.5cm]{logo/cs2c-logo}}
\renewcommand{\headrulewidth}{0.4pt}

% 行距
\linespread{1.6}

\lstset{
basicstyle=\ttfamily,
columns=flexible,
numbers=left,
numberstyle=\footnotesize,
%keywordstyle=\color{blue!70},
commentstyle=\color{red!50!green!50!blue!50},
frame=single,backgroundcolor=\color{white}, 
framexleftmargin=2em,
breaklines=true,
escapeinside=``,
xleftmargin=3.5em,
xrightmargin=1em, 
aboveskip=1em
}

% 正文
\begin{document}

%%%% cover %%%%
\thispagestyle{empty}

\noindent\begin{minipage}{\textwidth}
	\begin{flushleft}
	\includegraphics{logo/cs2c-short-logo}
	\end{flushleft}
\end{minipage}

\vspace{\stretch{1}}
\noindent\begin{minipage}{\textwidth}
\centering
{\LARGE \bfseries Shell~脚本执行的几种方式}
\noindent\rule[1.5ex]{\textwidth}{1pt}
\end{minipage}

\vspace{\stretch{2}}
\noindent\begin{center}
2016~年~07~月~16~日
\end{center}
\clearpage

%%%% main body %%%%
\section{Shell~脚本执行的方式}
bash\footnote{本文档以bash为例进行说明}~的命令分为两类:外部命令和内部命令。外部命令是通过系统调用或独立的程序实现的,如~sed、awk~等。内部命令是由特殊的文件格式(.def)所实现,如~cd、history、exec~等。fork是linux的系统调用,用来创建子进程。exec~和~source~都属于~bash~内部命令(builtins commands)。
\subsection{fork}
\textbf
{fork(/directory/script.sh):如果~shell~中包含执行命令,那么子命令并不影响父级的命令,在子命令执行完后再执行父级命令。子级的环境变量不会影响到父级。}

fork~是最普通的,就是直接在脚本里面用~/directory/script.sh~来调用~script.sh~这个脚本。运行的时候开一个~sub-shell~执行调用的脚本,sub-shell~执行的时候, parent-shell~还在。sub-shell~执行完毕后返回~parent-shell。 sub-shell~从~parent-shell~继承环境变量,但是~sub-shell~中的环境变量不会带回~parent-shell。

子进程是父进程(parent process)的一个副本,从父进程那里获得一定的资源分配以及继承父进程的环境。子进程与父进程唯一不同的地方在于~pid(process id)。

环境变量(传给子进程的变量,遗传性是本地变量和环境变量的根本区别)只能单向从父进程传给子进程。不管子进程的环境变量如何变化,都不会影响父进程的环境变量。

\subsection{exec}
\textbf
{exec(exec /directory/script.sh):执行子级的命令后,不再执行父级命令。}

exec~命令在执行时会把当前的~shell process~关闭,然后换到后面的命令继续执行。exec~与~fork~不同,不需要新开一个~sub-shell~来执行被调用的脚本。被调用的脚本与父脚本在同一个~shell~内执行。但是使用~exec~调用一个新脚本以后,父脚本中~exec~行之后的内容就不会再执行了。这是~exec~和~source~的区别。

\subsection{source/(.)}
\textbf
{source(source /directory/script.sh):执行子级命令后继续执行父级命令,同时子级设置的环境变量会影响到父级的环境变量。}
source命令即点(.)命令。

与fork的区别是不新开一个sub-shell来执行被调用的脚本,而是在同一个shell中执行。所以被调用的脚本中声明的变量和环境变量, 都可以在主脚本中得到和使用。

\section{代码说明参考}
以下代码取自于网络,但是能够很清晰地说明~Shell~脚本的几种执行方式。

1.sh
\begin{lstlisting}[language=bash,showstringspaces=false]
#!/bin/bash
A=B 
echo "PID for 1.sh before exec/source/fork:$$"
export A
echo "1.sh: \$A is $A"
case $1 in
        exec)
                echo "using exec…"
                exec ./2.sh ;;
        source)
                echo "using source…"
                . ./2.sh ;;
        *)
                echo "using fork by default…"
                ./2.sh ;;
esac
echo "PID for 1.sh after exec/source/fork:$$"
echo "1.sh: \$A is $A"
\end{lstlisting}

2.sh
\begin{lstlisting}[language=bash,showstringspaces=false]
#!/bin/bash
echo "PID for 2.sh: $$"
echo "2.sh get \$A=$A from 1.sh"
A=C
export A
echo "2.sh: \$A is $A"
\end{lstlisting}

执行情况:

\$ ./1.sh
\begin{lstlisting}[language=bash,showstringspaces=false]
PID for 1.sh before exec/source/fork:5845364
1.sh: $A is B
using fork by default…
PID for 2.sh: 5242940
2.sh get $A=B from 1.sh
2.sh: $A is C
PID for 1.sh after exec/source/fork:5845364
1.sh: $A is B
\end{lstlisting}

\$ ./1.sh exec
\begin{lstlisting}[language=bash,showstringspaces=false]
PID for 1.sh before exec/source/fork:5562668
1.sh: $A is B
using exec…
PID for 2.sh: 5562668
2.sh get $A=B from 1.sh
2.sh: $A is C
\end{lstlisting}

\$ ./1.sh source
\begin{lstlisting}[language=bash,showstringspaces=false]
PID for 1.sh before exec/source/fork:5156894
1.sh: $A is B
using source…
PID for 2.sh: 5156894
2.sh get $A=B from 1.sh
2.sh: $A is C
PID for 1.sh after exec/source/fork:5156894
1.sh: $A is C
\end{lstlisting}

\section{其他说明}
系统调用~exec~是以新的进程去代替原来的进程,但进程的~PID~保持不变。因此,可以这样认为,exec~系统调用并没有创建新的进程,只是替换了原来进程上下文的内容。原进程的代码段,数据段,堆栈段被新的进程所代替。

一个进程主要包括以下几个方面的内容:
\begin{enumerate}
\item 一个可以执行的程序;
\item 与进程相关联的全部数据(包括变量,内存,缓冲区);
\item 程序上下文(程序计数器PC,保存程序执行的位置)。
\end{enumerate}

exec是一个函数簇,由6个函数组成,分别是以excl和execv打头的。执行exec系统调用,一般都是这样,用fork()函数新建立一个进程,然后让进程去执行exec调用。我们知道,在fork()建立新进程之后,父进各与子进程共享代码段,但数据空间是分开的,但父进程会把自己数据空间的内容copy到子进程中去,还有上下文也会copy到子进程中去。而为了提高效率,采用一种写时copy的策略,即创建子进程的时候,并不copy父进程的地址空间,父子进程拥有共同的地址空间,只有当子进程需要写入数据时 (如向缓冲区写入数据),这时候会复制地址空间,复制缓冲区到子进程中去。从而父子进程拥有独立的地址空间。而对于fork()之后执行exec后,这种策略能够很好的提高效率,如果一开始就copy,那么exec之后,子进程的数据会被放弃,被新的进程所代替。

\end{document}