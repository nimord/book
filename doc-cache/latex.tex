\chapter{\LaTeX~学习}

\section{基础学习}

\subsection{软件准备}
\label{subsec:latexsoft}

\LaTeX~是一个软件系统,同时也是一套标准。遵照这些标准,实现了(implement)所要求功能的软件集合被称为发行版(distribution)。与此类似的例子有~Java~和~Linux,比如SUN、IBM、BEA~等公司都有自己的~Java~虚拟机(JVM),它们都被称作~Java~的实现;而Linux~有~Red Hat/~Fedora、Ubuntu、SuSE~等众多的发行版。

\begin{table}[htbp]
\caption{\LaTeX~发行版与编辑器}
\label{tab:latexsoft}
\centering
\begin{tabular}{lll}
    \toprule
    操作系统 & 发行版 & 编辑器 \\
    \midrule
    Windows & \href{http://www.miktex.org/}{MikTeX} & \href{http://www.toolscenter.org/}{TeXnicCenter}、\href{http://www.winedt.com/}{WinEdt} \\
    Unix/Linux & \href{http://www.tug.org/texlive/}{TeX Live} & \href{http://www.gnu.org/software/emacs/emacs.html}{Emacs}、\href{http://vim.sourceforge.net/}{vim}、\href{http://kile.sourceforge.net/}{Kile} \\
    Mac OS & \href{http://www.tug.org/mactex/}{MacTeX} & \href{http://www.uoregon.edu/~koch/texshop/}{TeXShop} \\
    \bottomrule
\end{tabular}
\end{table}

\LaTeX~发行版只提供了一个~\LaTeX~后台处理机制,用户还需要一个前台编辑器来编辑它的源文件。常用的~\LaTeX~发行版和编辑器见\ref{tab:latexsoft}。在使用~\LaTeX~的过程中可能还需要其它一些软件。

\subsection{学习方法}
\begin{quotation}
无他,唯手熟尔。
\begin{flushright}
---~卖油翁
\end{flushright}
\end{quotation}

\begin{quotation}
用心。
\begin{flushright}
---~斯蒂芬·周
\end{flushright}
\end{quotation}

比较严谨的入门资料有~Tobias Oetiker~的《A (Not So) Short Introduction to \LaTeXe》(简称lshort);若想对~\LaTeX~有更深入全面的了解,可以拜读~Mittelbach~的《The \LaTeX~ Companion》。

中文资料可参考李果正的《大家来学~\LaTeX》,lshort~有吴凌云等人翻译的中文版本。

\href{http://www.ctan.org/}{Comprehensive TeX Archive Network}(CTAN)和~\href{http://www.tug.org/}{TeX Users Group}~(TUG)提供了权威、丰富的资源。

\href{http://www.ctex.org}{CTeX}~分别提供了常见问题集(FAQ),一般问题多会在这里找到答案。

中文~\TeX~论坛有\href{http://www.smth.org/bbsdoc.php?board=TeX}{水木清华~BBS TeX~版}、\href{http://bbs.ctex.org/}{CTeX~论坛}。

\subsection{格式及其转换}
页面描述语言(Page Description Language,PDL)是一种在较高层次上描述实际输出结果的语言。本文只讨论其中三种与~\LaTeX~紧密相关的格式:DVI、PostScript、PDF。

\label{subsec:convert_format}
DVI、PS、PDF~等格式的的转换关系如\ref{fig:convert_format}所示。

\begin{figure}[htbp]
\centering
\begin{tikzpicture}
    \node[box] (tex) {.tex};
    \node[box] (dvi) [right=5 of tex] {.dvi};
    \node[box] (pdf) [right=6 of dvi] {.pdf};
    \node[box] (ps) [above=3 of pdf] {.ps};
    \path (tex) edge [arrow] node[auto] {latex} (dvi)
        (dvi) edge [arrow] node[auto] {dvipdfm} (pdf)
        (dvi.north) [arrow,draw] to node[above,sloped] {dvips} (ps.west)
        (ps) edge [arrow] node[right] {ps2pdf} (pdf)
        (tex) edge [arrow,bloop] (pdf);
    \node [below=.7 of dvi] {pdflatex};
\end{tikzpicture}
\caption{格式转换}
\label{fig:convert_format}
\end{figure}

最早的~driver~是~\verb|dvips|,它把~DVI~转换为~PS。\verb|dvipdf|~把~DVI~转为~PDF,它后来被~\verb|dvipdfm|~所取代;\verb|dvipdfm|~主要用于处理单字节字符,1999~年之后停止开发;在~\verb|dvipdfm|~基础上发展来的~\verb|dvipdfmx|~可以处理多字节编码。

pdf\TeX~是一种特殊的driver,它跳过~DVI,直接用~\TeX~源文件生成~PDF。基于~pdf\TeX~的~pdf\LaTeX~则把\LaTeX~源文件转为~PDF。

\verb|dvipdfmx|对图形格式的兼容性较好,而且擅长处理中文。

得到~DVI~后,我们可以在控制台用以下命令把它转为~PDF。
\begin{code}
dvipdfm xxx(.dvi)
\end{code}

我们也可以把它转为~PS,接着用~Ghostscript~的一个命令行程序把它转换为~PDF,注意第二个命令需要~\verb|.ps|~后缀。一般情况下不推荐这种方法,因为它多了个步骤。
\begin{code}
dvips xxx(.dvi)
ps2pdf xxx.ps
\end{code}

pdf\LaTeX~用法如下
\begin{code}
pdflatex xxx(.tex)
\end{code}

\subsection{\LaTeX~语句}
\LaTeX~源文件的每一行称作一条语句(statement),语句可以分三种:命令(command)、数据(data)和注释(comment)。

命令分为两种:普通命令和环境(environment)。普通命令以\verb|\|~起始,大多只有一行;而环境包含一对起始声明和结尾声明,用于多行的场合。命令和环境可以互相嵌套。

\LaTeX~源文件的结构分三大部分,依次为:文档类声明、序言(可选)、正文。
这三部分的基本语法如下:
\begin{code}
\documentclass[options]{class}  %文档类声明
\usepackage[options]{package}   %引入宏包
...
\begin{document}                %正文
...
\end{document}
\end{code}

把下面例子用编辑器保存为~\verb|hello_world.tex|,这就是一个最简单的~\LaTeX~源文件。
%\label{subsec:hello_world}
\begin{code}
%hello_world.tex
\documentclass{article}
\begin{document}
    Hello, World!
\end{document}
\end{code}

\section{\LaTeX~ 中文模版}
\label{sec:template}
下面是一个~\LaTeX~ 中文模版\footnote{这是我找到的模版中最好的一个了,尽管没有涉及到系统命令的更改}, 仅供参考!

\begin{verbatim}
%%%%%%%%%%%%%%%%%%%%%%%%%%%%%%%%%%%
\iffalse                             % 块注释
如果要注释一块文字,用\iffalse ... \fi 界定住要
注释的文字。特别提醒:以下设置的次序不能乱,否则
会引发冲突,影响到编译是否成功。
\fi
\documentclass[a4paper,11pt,         % A4纸
               twoside,              % 双面
%              openany               % 新章节在偶数页开始
               ]{article}

%%%%%%%%%% 版面控制 %%%%%%%%%%
\usepackage{indentfirst}             % 首行缩进
\iffalse
\usepackage[%paperwidth=18.4cm, paperheight= 26cm,
            body={14.6true cm,22true cm},
            twosideshift=0 pt,

 %headheight=1.0true cm
            ]{geometry}
\fi
\usepackage[perpage,symbol]{footmisc}% 脚注控制
\usepackage[sf]{titlesec}            % 控制标题
\usepackage{titletoc}                % 控制目录
\usepackage{fancyhdr}                % 页眉页脚
\usepackage{type1cm}                 % 控制字体大小
\usepackage{indentfirst}             % 首行缩进
\usepackage{makeidx}                 % 建立索引
\usepackage{textcomp}                % 千分号等特殊符号
\usepackage{layouts}                 % 打印当前页面格式
\usepackage{bbding}                  % 一些特殊符号
\usepackage{cite}                    % 支持引用
\usepackage{color,xcolor}            % 支持彩色文本、底色、文本框等
\usepackage{listings}                % 粘贴源代码
\lstloadlanguages{}                  % 所要粘贴代码的编程语言
\lstset{language=,tabsize=4, keepspaces=true,
    xleftmargin=2em,xrightmargin=2em, aboveskip=1em,
    backgroundcolor=\color{lightgray},    % 定义背景颜色
    frame=none,                      % 表示不要边框
    keywordstyle=\color{blue}\bfseries,
    breakindent=22pt,
    numbers=left,stepnumber=1,numberstyle=\tiny,
    basicstyle=\footnotesize,
    showspaces=false,
    flexiblecolumns=true,
    breaklines=true, breakautoindent=true,breakindent=4em,
    escapeinside={/*@}{@*/}
}

%%%%%%%%%% 字体支持 %%%%%%%%%%%%
%\usepackage{ccmap}                  % 使pdfLatex生成的文件支持复制等
\usepackage{CJK,CJKnumb,CJKulem}     % 中文支持
\usepackage{times}     % 包括 Times Roman + Helvetica + Courier
%\usepackage{palatino} % 包括 Palatino + Helvetica + Courier
%\usepackage{newcent} %包括 New Century Schoolbook + Avant Garde + Courier
%\usepackage{bookman}  % 包括 Bookman + Avant Garde + Courier

%%%%%%%%%% 数学符号公式 %%%%%%%%%%
\usepackage{latexsym}
\usepackage{amsmath}                 % AMS LaTeX宏包
\usepackage{amssymb}                 % 用来排版漂亮的数学公式
\usepackage{amsbsy}
\usepackage{amsthm}
\usepackage{amsfonts}
\usepackage{mathrsfs}                % 英文花体字体
\usepackage{bm}                      % 数学公式中的黑斜体
\usepackage{relsize}    % 调整公式字体大小:\mathsmaller, \mathlarger
\usepackage{caption2}                % 浮动图形和表格标题样式

%%%%%%%%%% 图形支持宏包 %%%%%%%%%%
\ifx\pdfoutput\undefined             % 用latex或pdflatex编译
  \usepackage[dvips]{graphicx}       % 将eps格式的图片放在figures目录下
\else                                % 在setup/format.tex中用以下命令注明路径:
  \usepackage[pdftex]{graphicx}      % \graphicspath{{figures/}}
\fi
%\usepackage{subfigure}
\usepackage{epsfig}                  % 支持eps图像
%\usepackage{picinpar}               % 图表和文字混排宏包
%\usepackage[verbose]{wrapfig}       % 图表和文字混排宏包
%\usepackage{eso-pic}                % 向文档的部分页加n副图形, 可实现水印效果
%\usepackage{eepic}                  % 扩展的绘图支持
%\usepackage{curves}                 % 绘制复杂曲线
%\usepackage{texdraw}                % 增强的绘图工具
%\usepackage{treedoc}                % 树形图绘制
%\usepackage{pictex}                 % 可以画任意的图形
%\usepackage{hyperref}

%%%%%%%%%% 一些距离设置 %%%%%%%%%%%
\setlength{\floatsep}{10pt plus 3pt minus 2pt}       % 图形之间或图形与正文之间的距离
\setlength{\abovecaptionskip}{2pt plus 1pt minus 1pt}% 图形中的图与标题之间的距离
\setlength{\belowcaptionskip}{3pt plus 1pt minus 2pt}% 表格中的表与标题之间的距

%%%%%%%%%% 纸张和页面的大小 %%%%%%%%%%
%\paperwidth   20 true cm            % 纸张宽
%\paperheight  30 true cm            % 纸张高
%\textwidth    10 true cm            % 正文宽
%\textheight   20 true cm            % 正文高
%\headheight      14pt               % 页眉高
%\headsep         16pt               % 页眉距离
%\footskip        27pt               % 页脚距离
%\marginparsep    10pt               % 边注区距离
%\marginparwidth  100pt              % 边注区宽
\makeindex                           % 生成索引
\pagestyle{fancy}                    % 页眉页脚风格
\fancyhf{}                           % 清空当前页眉页脚的默认设置

%%%%%%%%%% 导入中文环境 %%%%%%%%%%
\AtBeginDocument{\begin{CJK*}{GBK}{song} % 不计中文的空格
\CJKindent                           % 首行缩进两个汉字
\sloppy\CJKspace                     % 中英文混排的断行
\CJKtilde                            % 重新定义~,用~隔开中英文
\CJKcaption{GB}                      % 章节标题的中文化
}
\AtEndDocument{\end{CJK*}}

%%%%%%%%%% 正文 %%%%%%%%%%
\begin{document}

%%%%%%%%%% 一些新定义 %%%%%%%%%%
\newcommand{\song}{\CJKfamily{song}} % 宋体
\newcommand{\hei}{\CJKfamily{hei}}   % 黑体
\newcommand{\fs}{\CJKfamily{fs}}     % 仿宋
\newcommand{\kai}{\CJKfamily{kai}}   % 楷体

%%%%%%%%%% 定理类环境的定义 %%%%%%%%%%
%% 必须在导入中文环境之后
\newtheorem{example}{例}             % 整体编号
\newtheorem{algorithm}{算法}
\newtheorem{theorem}{定理}[section]  % 按 section 编号
\newtheorem{definition}{定义}
\newtheorem{axiom}{公理}
\newtheorem{property}{性质}
\newtheorem{proposition}{命题}
\newtheorem{lemma}{引理}
\newtheorem{corollary}{推论}
\newtheorem{remark}{注解}
\newtheorem{condition}{条件}
\newtheorem{conclusion}{结论}
\newtheorem{assumption}{假设}

%%%%%%%%%% 一些重定义 %%%%%%%%%%
%% 必须在导入中文环境之后
\renewcommand{\contentsname}{目录}     % 将Contents改为目录
\renewcommand{\abstractname}{摘\ \ 要} % 将Abstract改为摘要
\renewcommand{\refname}{参考文献}      % 将References改为参考文献
\renewcommand{\indexname}{索引}
\renewcommand{\figurename}{图}
\renewcommand{\tablename}{表}
\renewcommand{\appendixname}{附录}
\renewcommand{\proofname}{\hei 证明}
\renewcommand{\algorithm}{\hei 算法}

%%%%%%%%%% 重定义字号命令 %%%%%%%%%%
\newcommand{\yihao}{\fontsize{26pt}{36pt}\selectfont}       % 一号, 1.4倍行距
\newcommand{\erhao}{\fontsize{22pt}{28pt}\selectfont}       % 二号, 1.25倍行距
\newcommand{\xiaoer}{\fontsize{18pt}{18pt}\selectfont}      % 小二, 单倍行距
\newcommand{\sanhao}{\fontsize{16pt}{24pt}\selectfont}      % 三号, 1.5倍行距
\newcommand{\xiaosan}{\fontsize{15pt}{22pt}\selectfont}     % 小三, 1.5倍行距
\newcommand{\sihao}{\fontsize{14pt}{21pt}\selectfont}       % 四号, 1.5倍行距
\newcommand{\bansi}{\fontsize{13pt}{19.5pt}\selectfont}     % 半四, 1.5倍行距
\newcommand{\xiaosi}{\fontsize{12pt}{18pt}\selectfont}      % 小四, 1.5倍行距
\newcommand{\dawu}{\fontsize{11pt}{11pt}\selectfont}        % 大五, 单倍行距
\newcommand{\wuhao}{\fontsize{10.5pt}{10.5pt}\selectfont}   % 五号, 单倍行距

%%%%%%%%%% 页眉和页脚的设置 %%%%%%%%%%
\lhead{一个~\LaTeX+CJK~的简单模板}
\rhead{\TeX~爱好者}
\lfoot{用~\LaTeX~写科技论文}
\rfoot{~\thepage~}

%%%%%%%%%% 论文标题、作者等 %%%%%%%%%%
\title{用~\LaTeX~写科技论文            % 论文标题
      \thanks{这是一个为初学者写的~\LaTeX+CJK~论文模板,未经作者允许可以
      随意下载使用并修改传播,目的是让更多的人迅速上手用~\LaTeX~系统写作。}
       }
\author{于江生\\                     % 作者
        北京大学计算机系}
\date{2008年10月01日}                % 日期
\maketitle                           % 生成标题
\tableofcontents                     % 插入目录
\thispagestyle{empty}                % 首页无页眉页脚

\begin{abstract}
\noindent % 不缩进
\end{verbatim}

这是一个简单的~\LaTeX+CJK~的模板,为~\TeX~的初学者提供便利上手的参照。